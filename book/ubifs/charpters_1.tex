\chapter{UBIFS简介}
无序区块镜像文件系统(Unsorted Block Image File System, UBIFS)是用于固态存储设备上,并与LogFS相互竞争,作为JFFS2的后继文件系统之一。

UBIFS是专门为了解决MTD(Memory Technology Device)设备所遇到的性能瓶颈而设计的。由于Nand Flash容量的暴涨,YAFFS等皆无法操控大的Nand Flash空间。UBIFS通过子系统UBI处理与MTD device之间的动作。与JFFS2一样,UBIFS 建构于MTD device 之上,因而与一般的block device(例如: emmc, sd-card等)不兼容。

\section{UBIFS子系统}
JFFS2与UBIFS区别是,JFFS2是可以直接操作MTD设备,而UBIFS是工作在UBI卷上。因此对于UBIFS来说其实有三个子系统:
\begin{itemize}
  \item MTD子系统,它提供了访问flash的标准接口,并向上层应用程序提供类似/dev/mtd[0-99]接口
  \item UBI子系统,它为flash提供了损耗均衡与卷管理系统,UBI是MTD的高层次表示,为UBIFS处理flash的坏块管理与损耗均衡等工作
  \item UBIFS文件系统,工作在UBI卷上
\end{itemize}


