\chapter{UBIFS简介}
无序区块镜像文件系统(Unsorted Block Image File System, UBIFS)是用于固态存储设备上,并与LogFS相互竞争,作为JFFS2的后继文件系统之一。

UBIFS是专门为了解决MTD(Memory Technology Device)设备所遇到的性能瓶颈而设计的。由于Nand Flash容量的暴涨,YAFFS等皆无法操控大的Nand Flash空间。UBIFS通过子系统UBI处理与MTD device之间的动作。与JFFS2一样,UBIFS 建构于MTD device 之上,因而与一般的block device(例如: emmc, sd-card等)不兼容。

\section{UBI子系统}
JFFS2与UBIFS区别是,JFFS2是可以直接操作MTD设备,而UBIFS是工作在UBI卷上。因此对于UBIFS来说其实有三个子系统:
\begin{itemize}
  \item MTD子系统,它提供了访问flash的标准接口,并向上层应用程序提供类似/dev/mtd[0-99]接口
  \item UBI子系统,它为flash提供了损耗均衡与卷管理系统,UBI是MTD的高层次表示,为UBIFS处理flash的坏块管理与损耗均衡等工作
  \item UBIFS文件系统,工作在UBI卷上
\end{itemize}

UBI卷,与分区概念类似,一个分区是一组连续的物理地址的集合,而卷是一组连续的逻辑地址集合,所以卷中的擦除块又称为逻辑擦除块。这个逻辑块可以映射到分区上任一个物理块上。UBI有两种卷类型,可读写的动态卷与只读的静态卷。

UBI卷层(layout volume),卷层是一个特殊的卷,称为内部卷,它里面包含了ubi的卷表信息。卷层包含两个逻辑擦除块,其中一个卷是为了备份另一个卷而存在的。

UBI坏块管理,UBI的存在使得UBIFS不用关注坏块问题。UBI会在flash中预留一部分的冗余块,这些块是用来替换使用过程中出现的坏块。UBI将旧块数据移动到替换块上,然后改变UBI卷中的逻辑块到物理块的映射关系。

UBI冲刷机制,因为ECC的存在,flash少量的bit-flip是不会导致系统出现问题的,但是bit-flip累积到一定数量就会导致系统出现问题。UBI冲刷是针对出现bit-flip的块,把待冲刷块数据搬迁到其它块上,然后对这个块进行擦除操作,以确定它是不是坏块。

UBI损耗均衡,UBI会记录每一个块擦除次数,对于擦除次数过多的块,会把它的数据移到擦除次数少的块,然后改变卷的逻辑块到物理块的映射。

