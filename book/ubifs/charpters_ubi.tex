\chapter{UBI子系统分析}
\section{UBI ATTACH}
在使用ubifs文件系统之前,需要先在内核中安装ubi模块,然后将ubifs镜像文件attach到ubi中,然后执行挂载操作。为什么ubifs不能像ext2等文件系统一样直接挂载使用呢?因为ubifs不是建立在flash基础上,而是建立在ubi子系统上的,因此在使用ubifs前需要先让镜像与ubi建立关系。ubi子系统不仅可以供ubifs使用,也能提供给其它文件系统使用。

\subsection{ubi重要数据结构}
\begin{lstlisting}[language=C]
/*
 * ubi_device是在UBI做ATTACH时建立的,代表一个MTD分区。一个ubi_device下面可以存在多个卷
 */
struct ubi_device {
	struct cdev cdev; // 类似/dev/ubi0
	struct device dev; //ubi设备驱动模型
	int ubi_num; //ubi编号(/dev/ubi0就是0)
	char ubi_name[sizeof(UBI_NAME_STR)+5]; //ubi设备名,一般就是ubi[0..99]
	int vol_count; //这个ubi设备有多少个卷(外部卷+内部卷)
	struct ubi_volume *volumes[UBI_MAX_VOLUMES+UBI_INT_VOL_COUNT]; //卷指针数组,用于查找卷
	spinlock_t volumes_lock; //用于保护ubi设备的引用计数等临界数据
	int ref_count; //ubi设备引用计数,ref_count不为0代表有进程正在使用ubi设备,此时不能删除ubi设备
	int image_seq; //顺序号,主要用于一个LEB映射到多个PEB的情况
	int rsvd_pebs; //该ubi设备所有卷占用的擦除块数量之和(包括冗余块在内)
	int avail_pebs; //初始化时使用,初始值为分区块数量减去坏块数量,然后使用这个值去分配其它块数量
	int beb_rsvd_pebs; //实际为坏块保留的擦除块数量(如果空闲块充足,它与beb_rsvd_level相等)
	int beb_rsvd_level; //正常情况下应该为坏块保留的擦除块数量(冗余块)
	int bad_peb_limit; //一个ubi设备最大坏块数量限制,超过这个数量,ubi设备将不能使用
	int autoresize_vol_id; //这个参数为autosize卷准备的,一个ubi设备最多只有一个autosize卷
	int vtbl_slots; //该ubi设备支持的卷的数量
	int vtbl_size; //vtbl_slots * sizeof(struct ubi_vtbl_record)
	struct ubi_vtbl_record *vtbl; //ubi设备的卷记录表(从卷层中读出)
	struct mutex device_mutex;
	int max_ec; //记录ubi设备最大擦除计数
	int mean_ec; //平均擦除计数
	unsigned long long global_sqnum; //全局顺序号,不停在增加(一般写LEB头部时增加)
	spinlock_t ltree_lock;
	struct rb_root ltree;
	struct mutex alc_mutex;
	/* 这些红黑树上都是挂的ubi_wl_entry结构 */
	struct rb_root used; //已使用的物理擦除块红黑树
	struct rb_root erroneous; //在搬迁数据时发生读错误的块会被挂在这个树上
	struct rb_root free; //被擦除后未使用的块,可以直接写入数据
	int free_count; //free块数量
	struct rb_root scrub; //发生了可纠正的bit-flip错误会被挂在擦洗树中
	/* 搬迁数据时卷被删除或待搬迁数据的LEB被释放时,把原先与LEB映射的PEB挂入保护链表 */
	struct list_head pq[UBI_PROT_QUEUE_LEN]; 
	int pq_head;
	spinlock_t wl_lock; //WL子系统的work锁,一般用在唤醒,添加,删除work时
	struct mutex move_mutex;
	struct rw_semaphore work_sem;
	int wl_scheduled; //非0表示WL work被调度,用来阻止重复调度
	struct ubi_wl_entry **lookuptbl; //查询wl_entry的表,用pnum当参数(每个物理块都有这个结构)
	struct ubi_wl_entry *move_from; //数据从一个物理块搬迁到另一个物理块时使用
	struct ubi_wl_entry *move_to; //同上
	int move_to_put;
	struct list_head works;
	int works_count;
	struct task_struct *bgt_thread;//ubi设备的后台进程,一般启动擦除块,scrub块等工作 
	int thread_enabled;//为0代表不唤醒ubi设备的后台进程
	char bgt_name[sizeof(UBI_BGT_NAME_PATTERN)+2];//后台进程名,ps会显示出来
	bool scrub_in_progress;
	atomic_t scrub_work_count;
	/* I/O sub-system's stuff */
	long long flash_size;
	int corr_peb_count; //损坏块数量,不再参与损耗平衡
	int erroneous_peb_count; //数据搬迁时,待搬迁块发生读错误(非bit-flip)
	int max_erroneous; //最大允许的错误块,一般为整个PEB的10%
	int min_io_size;
	int hdrs_min_io_size;
	int ro_mode;
	int leb_size; //LEB = PEB - 头部(ec与vid)
	int leb_start; //LEB_OFFSET = PEB_ADDR - 头部SIZE
	int ec_hdr_alsize;
	unsigned int bad_allowed:1;
	unsigned int nor_flash:1; //norflash与nandflash标志
	int max_write_size; //从mtd的writebufsize中获取,一般为一个pagesize
	struct mtd_info *mtd;//指向一个MTD分区
	void *peb_buf; //一个PEB大小的缓冲区,一般用在后台进程擦除块,做块擦洗动作
	struct mutex buf_mutex;
	struct mutex ckvol_mutex;
	struct ubi_debug_info dbg;
};
\end{lstlisting}

\begin{lstlisting}[language=C]
/* UBI卷描述符,里面大量信息由卷层中的信息填充(ubi_read_volume_table),
 * attach时由ubi_ainf_volume等信息来填充
 */
struct ubi_volume {
	struct device dev; //它的父设备是ubi_device的dev
	struct cdev cdev; //类似于/dev/ubi0_0
	struct ubi_device *ubi; //通过卷信息找到它所属的ubi设备
	int vol_id; //卷ID,在attach时由ubi_ainf_volume赋值
	int ref_count; //引用计数
	int readers; //以读方式持有该卷的进程数量
	int writers; //以读写方式持有该卷的进程数量
	int exclusive; //该卷是否被独占持有的标志
	int metaonly;
	int reserved_pebs; //该卷保留的PEB数量(制作UBI镜像时指定的,从卷表中读出)
	int vol_type; //动态卷与静态卷
	int usable_leb_size; //leb size
	int used_ebs;
	int last_eb_bytes;
	long long used_bytes;
	int alignment;
	int data_pad;
	int name_len; //卷名长度
	char name[UBI_VOL_NAME_MAX + 1]; //卷名
	int upd_ebs;
	int ch_lnum;
	long long upd_bytes;
	long long upd_received;
	void *upd_buf;
	int *eba_tbl; //用来建立卷的peb与leb的映射关系
	unsigned int checked:1;
	unsigned int corrupted:1;
	unsigned int upd_marker:1;
	unsigned int updating:1;
	unsigned int changing_leb:1;
	unsigned int direct_writes:1;
};
\end{lstlisting}

\begin{lstlisting}[language=C]
/* MTD设备attach信息,attach的信息会用在ubi的wl,eba,io等子系统上
 * 仅在attch时使用,attach结束后释放
 */
struct ubi_attach_info {
	struct rb_root volumes; //一个ubi设备上的卷挂在这里
	struct list_head corr; //一个ubi设备上所有损坏块都挂在这里,用于wl初始化
	struct list_head free; //一个ubi设备上所有空闲块
	struct list_head erase; //attach时发现的需要擦除的块(包括由power cut造成的损坏块)
	struct list_head alien; //不兼容块,指内部卷的块
	int corr_peb_count;
	int empty_peb_count; //EC头为0XFF的数量,空块会挂在erase链表上,会重新初始化
	int alien_peb_count;
	int bad_peb_count; //坏块数量
	int maybe_bad_peb_count; //可能是坏块,但是在OOB中没有坏块标记
	int vols_found; //已发现卷的数量
	int highest_vol_id;
	int is_empty;
	int min_ec; //最小擦除次数(从scand_peb时更新)
	int max_ec; //最大擦除数
	unsigned long long max_sqnum; //该ubi设备目前sqnum最大值,sqnum在PEB头部中
	int mean_ec; //所有块的平均擦除次数
	uint64_t ec_sum; //所有块总的擦除次数
	int ec_count;
	struct kmem_cache *aeb_slab_cache; //定义一个slab
};
\end{lstlisting}

\begin{lstlisting}[language=C]
/* attach关于卷信息填写,仅在attach时使用,attach结束后释放掉*/
struct ubi_ainf_volume {
	int vol_id; //卷ID
	int highest_lnum; //该卷最大逻辑编号,从0开始
	int leb_count;
	int vol_type; //动态卷与静态卷
	int used_ebs; //静态卷专有
	int last_data_size;
	int data_pad;
	int compat; //内部卷使用
	struct rb_node rb; //通过rb node挂载到ubi_attach_info的volumes中
	struct rb_root root; //上面挂载的是ubi_ainf_peb信息(包含一个卷所有的物理擦除块)
};
\end{lstlisting}

\begin{lstlisting}[language=C]
/* ubi_ainf_peb记录一个物理擦除块的信息,同一块擦除块可能有多个多个ubi_ainf_peb,一个
 * 通过rb node挂在ubi_ainf_volume的rb root中,其它的通过list挂入ubi_attach_info中
 * attach流程结束后,会释放所有的ubi_ainf_peb结构的内存
 */
struct ubi_ainf_peb {
	int ec; //该PEB的擦除次数
	int pnum; //PEB编号
	int vol_id; //PEB所属卷的ID
	int lnum; //PEB逻辑映射编号
	unsigned int scrub:1; //是否需要擦洗状态标志
	unsigned int copy_flag:1;
	unsigned long long sqnum;
	union {
		struct rb_node rb; //用rb node挂在ubi_ainf_volume的红黑树上
		struct list_head list; //用list挂在ubi_attach_info的对应链表上
	} u;
};
\end{lstlisting}

\clearpage

\subsection{UBI Attach流程}
UBI attach是为了建立UBI层与MTD分区连接,具体工作需要填充ubi\_device与ubi\_volume结构体,启动UBI WL子系统与EBA子系统,创建UBI层的用户层接口,创建后台工作线程处理flash擦除与写入工作。
UBI attach过程中还需要分配冗余块,每1024个块最多有多少坏块由宏CONFIG\_MTD\_UBI\_BEB\_LIMIT决定(一般为20),而这个限制值减去已标记坏块数量就是冗余块数量(如果当前空闲块不足,则以当前剩余块为准)
在attach过程中需要扫描每个PEB块,每个PEB块都有一个EC头部与VID头部(EC记录块擦除记数信息,VID记录卷ID信息),LEB就是PEB去了头部后内容。下面简单分析一下attach流程
\begin{mdframed}[style=leftredline]
\begin{verbatim}
ubi_attach_mtd_dev() //ubi attach起始函数,内核可以直接调用,用户空间通过/dev/ubi_ctrl调用
|--io_init() //初始化ubi_device的IO参数
|--ubi_attach() //初始化ubi所需的数据,ubi信息进入内存
|  |--scan_all() //扫描一个MTD分区所有的PEB块
/*
 * scan_peb会读取PEB块的头部,会根据头部信息创建UBI卷attach信息ubi_ainf_volume,
 * scan_peb会比较VID头部sqnum信息,如果有两个PEB具有相同的LEB逻辑号,则会把sqnum
 * 小的PEB挂到erase链表中(前提是sqnum大的不出出CRC较验错误),scan_peb还会较验每
 * 块EC,VID,data等数据,然后将它们挂在不同的链表中。
 */
|  |  |--scan_peb() //具体的块扫描函数
|  |  |  |--ubi_add_to_av() //将PEB加到对应ubi_ainf_volume中
|  |--ubi_read_volume_table() //从卷层中读出卷表信息,ubi_device的vtbl指向它
|  |--ubi_wl_init() //初始化并启动UBI WL子系统(它有一个内核线程去做擦除与负载均衡的操作)
|  |--ubi_eba_init() //初始化UBI EBA子系统,EBA管理着LEB与PEB的映射,以及封装UBI的读写接口
|  |  |--ubi_calculate_reserved() //计算需要为坏块管理保留多少冗余块
|  |--destory_ai() //清除attach时创建的数据,包括ubi_ainf_peb,ubi_ainf_volume等
\end{verbatim}
\end{mdframed}
UBI attach过程会把每个PEB块都进行分类,分类如下:
\begin{itemize}
  \item free,空闲块
  \item used,正常使用块
  \item corr,损坏块,不再使用
  \item alien,异常块,不再使用
  \item scrub,擦洗块,块上数据需要搬迁到其它块上,然后对其进行erase操作,以确认它是否是坏块
  \item torture,拷问块,块上数据需要搬迁到其它块上,然后对其反复erase操作,以确认它是否是坏块
\end{itemize}
块的分类是根据scan\_peb时出现的错误进行分类的,错误如下几类:
\begin{itemize}
  \item UBI\_IO\_FF(UBI\_IO\_FF\_BITFLIPS),表示读出的数据只有0XFF,BITFLIPS表示出现了可较验的位翻转
  \item UBI\_IO\_BAD\_HDR,EC或VID出现损坏(magic不匹配或CRC较验失败)
  \item UBI\_IO\_BAD\_HAD\_EBADMSG,EC或VID出现损坏(MTD上报的无法较验的ECC错误)
  \item UBI\_IO\_BITFLIPS,读数据出现可较验的位翻转
\end{itemize}
\clearpage
\begin{tabu} to \hsize {|X[3.5,l]|X[3.5,l]|X[2,l]|X|}
\hline
ubi\_ec\_hdr(ERR) & ubi\_vid\_hdr(ERR) & data(ERR) & PEB\_LIST\\ \hline
TRUE & TRUE & TRUE & used\\ \hline
UBI\_IO\_FF 或
UBI\_IO\_FF\_BITFLIPS & X & X & erase\\ \hline
UBI\_IO\_BAD\_HDR 或
UBI\_IO\_BAD\_HDR\_EBADMSG & UBI\_IO\_BAD\_HDR 或
UBI\_IO\_BAD\_HDR\_EBADMSG & X & erase
(POWER CUT)\\ \hline
UBI\_IO\_BAD\_HDR 或
UBI\_IO\_BITFLIPS 或
UBI\_IO\_BAD\_HDR\_EBADMSG & UBI\_IO\_FF & X & erase\\ \hline
UBI\_IO\_BITFLIPS & X & X & scrub\\ \hline
TRUE & UBI\_IO\_FF & X & free\\ \hline
TRUE & UBI\_IO\_BAD\_HDR 或
UBI\_IO\_BAD\_HDR\_EBADMSG & EBADMSG 或
UBI\_IO\_FF 或
UBI\_IO\_BITFLIPS & erase
(POWER CUT)\\ \hline
TRUE & UBI\_IO\_BAD\_HDR 或
UBI\_IO\_BAD\_HDR\_EBADMSG & TRUE & corr\\ \hline
X & BIT\_IO\_BITFLIPS 或
UBI\_IO\_FF\_BITFLIPS & X & scrub\\ \hline
X & X & BIT\_IO\_BITFLIPS & scrub\\ \hline
X & VOL\_ID异常 且 
UBI\_COMPAT\_DELETE & X & erase\\ \hline
X & VOL\_ID异常 且 
UBI\_COMPAT\_PRESERVE & X & alien\\ \hline
X & VOL\_ID异常 且 
UBI\_COMPAT\_RO & X & ro\\ \hline
\end{tabu}

