\documentclass[a4paper,oneside,18pt]{book}
\usepackage[pagestyles]{titlesec}
\usepackage{xeCJK}
\usepackage{fontspec}
\usepackage[top=2.0cm,bottom=2.0cm,left=2.0cm,right=2.0cm]{geometry}
\usepackage{xcolor}
\usepackage{color}
\usepackage{mdframed}
\usepackage{listings}
\usepackage{caption}
\usepackage{fancybox}
\usepackage{indentfirst}
\usepackage{dirtree}
\usepackage[titles]{tocloft}
\usepackage{hyperref}
\usepackage{tabu}
\usepackage{float}
\usepackage{setspace}
\usepackage{fancyhdr}
\usepackage{lastpage}
\usepackage{booktabs} %调整表格线与上下内容的间隔 
\usepackage[contents=QUECTEL,color=red!30]{background}%添加水印
\usepackage{xcolor}%定义了一些颜色  
\usepackage{colortbl,booktabs}%第二个包定义了几个*rule  
%\usepackage[pagestyles]{titlesec}%设置章节标题格式,居中(center)
%\setmainfont{Arial}
\setmainfont{Times New Roman}%设置英文主字体
\setCJKmainfont{AR PL UKai TW MBE}%设置中文主字体
\setmonofont[Mapping={}]{SimSun}%设置等宽字体
\setCJKsansfont{Arial}

\renewcommand{\baselinestretch}{1.2}%设置行距,1.2倍行距
\titleformat{\chapter}{\Huge\bfseries}{第\thechapter 章}{1em}{}%command=chapter,format=\centring...,label=第\thechapter...其它忽略
\titlespacing*{\chapter}{0pt}{\baselineskip}{\baselineskip}%设置章下面的空白
\titleformat{\section}{\LARGE\bfseries}{\thesection }{1em}{}%设置节字体大小,属性
\titleformat{\subsection}{\Large\bfseries}{\thesubsection }{1em}{}%设置节字体大小,属性

\newcommand{\chuhao}{\bf\fontsize{42pt}{\baselineskip}\selectfont}
\newcommand{\sizefont}{\bf\fontsize{15pt}{\baselineskip}\selectfont}
\newcommand{\xiaoyi}{\bf\fontsize{24pt}{\baselineskip}\selectfont}
\newcommand{\xiaoer}{\bf\fontsize{18pt}{\baselineskip}\selectfont}
\newcommand{\shiyi}{\bf\fontsize{11pt}{\baselineskip}\selectfont}

\newcommand{\xiaosibf}{\bf\fontsize{12pt}{\baselineskip}\selectfont}
\newcommand{\xiaosi}{{\fontsize{12pt}{\baselineskip}\selectfont}}

\newcommand{\wordred}[1]{{\color{red}{#1}}}
\newcommand{\wordyellow}[1]{{\color{yellow}{#1}}}
\fancypagestyle{plain}{
  \pagestyle{fancy}
}

\hypersetup{colorlinks=true,linkcolor=black,citecolor=black,urlcolor=blue}%设置url颜色,关闭引用的红色方框

%在这里设置页眉,页脚
\pagestyle{fancy}
\lhead{\includegraphics[keepaspectratio,width=0.3\textwidth,height=0.25\textheight]{title/img/quectel_title.png}}%在页眉处插入图片
\chead{} 
\rhead{\textbf{LTE Module Series\\xxxxxx}} 
\lfoot{\textbf{xxxxxx}} 
\cfoot{\textbf{Confidential / Released}}%设置页脚的页码显示形式(2/4)
\rfoot{\thepage/\pageref{LastPage}}
\renewcommand{\headrulewidth}{2pt}%设置页眉线宽度 
\renewcommand{\footrulewidth}{1pt}%设置页脚线宽度
\setlength{\topskip}{0cm}%页眉与正文距离
\setlength{\skip\footins}{3cm}


%设置C语言代码框
\lstset{
keywordstyle=\color{blue!70},commentstyle=\color{red!50!green!50!blue!50},
frame=lrb,
basicstyle=\footnotesize\ttfamily,
rulesepcolor=\color{red!20!green!20!blue!20},
tabsize=4,%设置tab键空格数
showstringspaces=false,
escapeinside=@@,%设置逃逸字串,用来显示代码框的中文
belowcaptionskip=-1pt,
xleftmargin=8pt,
framexleftmargin=8pt,
framexrightmargin=5pt,
framextopmargin=5pt,
framexbottommargin=5pt,
framesep=0pt,
rulesep=0pt,
}
\DeclareCaptionFont{white}{\color{white}}
\DeclareCaptionFormat{listing}{\colorbox{gray}{\parbox{\textwidth}{#1#2#3}}}
\captionsetup[lstlisting]{format=listing,labelfont=white,textfont=white}


%设置框类型,单边,灰色
\mdfsetup{skipabove=\topskip,skipbelow=\topskip}
\global\mdfdefinestyle{leftredline}{
linecolor=gray,
linewidth=3pt,
topline=false,
bottomline=false,
rightline=false,
leftmargin=5mm,
font=\footnotesize,
}

%设置框类型,双边,灰色
\global\mdfdefinestyle{bothredline}{
linecolor=gray,%线条颜色
linewidth=3pt,%线宽
topline=false,%不显示方框上方线条
bottomline=false,%不显示方框底部线条
leftmargin=0cm,
backgroundcolor=gray!10,%加入背景色,灰色,阶度10
font=\footnotesize,
}

%设置代码目录树
\renewcommand*\DTstylecomment{\rmfamily\textsc}
\renewcommand*\DTstyle{\ttfamily\textcolor{black}}

%目录设置

\definecolor{ubuntured}{RGB}{44,0,30}

\begin{document}
\begin{titlepage}
\begin{figure}[htbp]
\begin{flushright}
\includegraphics[keepaspectratio,width=0.45\textwidth,height=0.55\textheight]{title/img/quectel_title.png}
\end{flushright}
\end{figure}

\vspace*{4cm}
{\chuhao{XXXXXX XXXXXX}}
\vspace*{2cm}

\renewcommand\arraystretch{2.0}%调整表格行间距
\begin{tabular}{l l }
{\sizefont{Author}} &{\sizefont{Darren}}\\
{\sizefont{Rivision}} &{\sizefont{V0.1}}\\
{\sizefont{Date}} &{\sizefont{2017/09/12}}\\
\end{tabular}

\vspace*{5cm}
\begin{figure}[htbp]
\includegraphics[keepaspectratio,width=\textwidth,height=0.75\textheight]{title/img/quectel.png}
\end{figure} 
\vspace*{1cm}
\begin{flushright}
www.quectel.com
\end{flushright}

\end{titlepage}
\vspace*{1.6cm}
\noindent
{\xiaosibf
Our aim is to provide customers with timely and comprehensive service. For any\\
assistance, please contact our company headquarters:}

\vspace{\baselineskip}
\noindent
{\xiaosibf
Quectel Wireless Solutions Co., Ltd.}\\
Office 501, Building 13, No.99, Tianzhou Road, Shanghai, China, 200233\\
Tel: +86 21 5108 6236\\
Email: \underline{info@quectel.com}\\

\vspace{\baselineskip}
\noindent
{\xiaosibf
Or our local office. For more information, please visit:}\\
\underline{http://www.quectel.com/support/salesupport.aspx}\\

\vspace{\baselineskip}
\noindent
{\xiaosibf
For technical support, or to report documentation errors, please visit:}\\
\underline{http://www.quectel.com/support/techsupport.aspx}\\
Or Email to: \underline{Support@quectel.com}

\vspace{\baselineskip}
\noindent
{\xiaosibf
GENERAL NOTES}\\
QUECTEL OFFERS THE INFORMATION AS A SERVICE TO ITS CUSTOMERS. THE INFORMATION\\
PROVIDED IS BASED UPON CUSTOMERS’ REQUIREMENTS. QUECTEL MAKES EVERY EFFORT\\
TO ENSURE THE QUALITY OF THE INFORMATION IT MAKES AVAILABLE. QUECTEL DOES NOT\\
MAKE ANY WARRANTY AS TO THE INFORMATION CONTAINED HEREIN, AND DOES NOT ACCEPT\\
ANY LIABILITY FOR ANY INJURY, LOSS OR DAMAGE OF ANY KIND INCURRED BY USE OF OR\\
RELIANCE UPON THE INFORMATION. ALL INFORMATION SUPPLIED HEREIN IS SUBJECT TO\\
CHANGE WITHOUT PRIOR NOTICE.

\vspace{\baselineskip}
\noindent
{\xiaosibf
COPYRIGHT}\\
THE INFORMATION CONTAINED HERE IS PROPRIETARY TECHNICAL INFORMATION OF\\
QUECTEL CO., LTD. TRANSMITTING, REPRODUCTION, DISSEMINATION AND EDITING OF THIS\\
DOCUMENT AS WELL AS UTILIZATION OF THE CONTENT ARE FORBIDDEN WITHOUT\\
PERMISSION. OFFENDERS WILL BE HELD LIABLE FOR PAYMENT OF DAMAGES. ALL RIGHTS\\
ARE RESERVED IN THE EVENT OF A PATENT GRANT OR REGISTRATION OF A UTILITY MODEL\\
OR DESIGN.

\vspace{\baselineskip}
\vspace{\baselineskip}
\noindent
{\xiaosibf
\emph{Copyright © Quectel Wireless Solutions Co., Ltd. 2017. All rights reserved.}}
\clearpage

\vspace*{2.5cm}
{\xiaoyi{About the Document}}
\vspace*{2cm}

{\xiaoer{History}}
\renewcommand\arraystretch{2}%设置表格宽度
\begin{table}[htbp]\large
\begin{tabular}{p{0cm}p{3cm}p{4cm}p{3cm}p{6cm}}%设置表格字段长度
\toprule[0pt] 
\arrayrulecolor{gray}%设置表格实线颜色
\rowcolor[gray]{0.9}  
&\bf{Revision} &\bf{Date}   &\bf{Author}  &\bf{Description}\\   
&0.1 &September 12,2017  &Darren  &Initial\\ 
\bottomrule[0.5pt]  
\end{tabular}  
\end{table}

\clearpage

\input{ch1.tex}
\end{document}
